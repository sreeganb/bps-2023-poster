\documentclass[final]{beamer}

% ====================
% Packages
% ====================

\usepackage[T1]{fontenc}
\usepackage{lmodern}
\usepackage[size=custom,width=120,height=72,scale=1.0]{beamerposter}
\usetheme{gemini}
\usecolortheme{uchicago}
\usepackage{graphicx}
\usepackage{booktabs}
\usepackage{doi}
\usepackage[numbers]{natbib}
\usepackage[patch=none]{microtype}
\usepackage{tikz}
\usepackage{pgfplots}
\pgfplotsset{compat=1.18}
\usepackage{anyfontsize}
\usepackage{subfig}

\pdfstringdefDisableCommands{%
\def\translate#1{#1}%
}

% ====================
% Lengths
% ====================

% If you have N columns, choose \sepwidth and \colwidth such that
% (N+1)*\sepwidth + N*\colwidth = \paperwidth
\newlength{\sepwidth}
\newlength{\colwidth}
\setlength{\sepwidth}{0.025\paperwidth}
\setlength{\colwidth}{0.3\paperwidth}

\newcommand{\separatorcolumn}{\begin{column}{\sepwidth}\end{column}}

% ====================
% Title
% ====================

\title{Elucidation of the mechanism of a Morita-Baylis-Hillman type reaction catalyzed by engineered enzymes using transition path sampling}

\author{Sree Ganesh Balasubramani \inst{1} \and Steven D. Schwartz \inst{1} }

\institute[shortinst]{\inst{1} Department of Chemistry and Biochemistry, The University of Arizona}

% ====================
% Footer (optional)
% ====================

\footercontent{
  \href{https://www.example.com}{https://sreeganb.github.io} \hfill
  %ABC Conference 2025, New York --- XYZ-1234 
  \hfill
  \href{mailto:sreegb@arizona.edu}{sreegb@arizona.edu}}
% (can be left out to remove footer)

% ====================
% Logo (optional)
% ====================

% use this to include logos on the left and/or right side of the header:
% \logoright{\includegraphics[height=7cm]{logo1.pdf}}
% \logoleft{\includegraphics[height=7cm]{logo2.pdf}}

% ====================
% Body
% ====================

\begin{document}
\addtobeamertemplate{headline}{}
{
    \begin{tikzpicture}[remember picture,overlay]
      \node [anchor=north west, inner sep=3cm] at ([xshift=0.0cm,yshift=-2.4cm]current page.north west)
      {\includegraphics[height=5.0cm]{logos/ua_stack_rgb_4.eps}}; % also try shield-white.eps
      %\node [anchor=north east, inner sep=3cm] at ([xshift=0.0cm,yshift=2.5cm]current page.north east)
      %{\includegraphics[height=8.0cm]{logos/cs-logo-white.png}};
    \end{tikzpicture}
}

\begin{frame}[t]
\begin{columns}[t]
\separatorcolumn

\begin{column}{\colwidth}

\begin{block}{Directed evolution and the design of enzymes}
Engineered enzymes to perform the complex Morita–Baylis–Hillman (MBH) reaction \cite{Crawshaw22NatChem14p313} 
    \begin{itemize}
     % \item Is it possible to design enzymes that perform complicated reactions such as this easily?
      \item Elucidate the mechanism and transition state of the double proton transfer reaction 
      %\item Find the transition state and propose mutations that improve the catalytic efficiency
      \item Understand the importance of quantum mechanical tunneling in evolutionary selection
    \end{itemize}
 
  \end{block}

\begin{block}{Molecular dynamics simulations}
\begin{itemize}
\item Atomistic simulations to understand chemical reactions catalyzed by enzymes.
\item Partition the enzyme into regions treated using classical molecular mechanics (MM)
and quantum mechanics (QM)
\item A density functional approximation called SCCDFTB is the choice for the QM method
\end{itemize}

\begin{figure}
\subfloat{\includegraphics[width=0.55\colwidth]{figures/mbhase.png}}
\quad
\subfloat{\includegraphics[width=0.38\colwidth]{figures/qmmm.png}}
\end{figure}
\end{block}

  \begin{block}{Introduction to transition path sampling (TPS)}
\begin{itemize}
    \item Enhanced sampling technique
    \item Monte-Carlo type method in the trajecotry space \cite{Bolhuis02AnnRevPhysChem53p291}
    \item Elucidate the transition state, reaction coordinates, thermodynamic and kinetic parameters \cite{Frost23JPhysChemB127p144}
    %\item Shooting and shifting algorithm
\end{itemize}
\begin{figure}%
\centering
\subfloat[Trajectories in phase space]{\includegraphics[width=0.45\colwidth]{figures/pot-surf.pdf}}%
\qquad
\subfloat[Shooting algorithm]{\includegraphics[width=0.45\colwidth]{figures/tps-desc.png}}
%\caption{Lets see}%
\label{fig:tps-desc}%
\end{figure}

  \end{block}

\end{column}

\separatorcolumn

\begin{column}{\colwidth}

  \begin{block}{Reactant and product states}
  \begin{figure}%
\centering
\subfloat[Reactant state]{\includegraphics[width=0.45\colwidth]{figures/reac-120.png}}%
\quad
\subfloat[Product state]{\includegraphics[width=0.45\colwidth]{figures/prod-120.png}}
%\caption{Lets see}%
\label{fig:tps-states}%
\end{figure}
   The probability distribution of the TPS trajectories is
   \begin{equation}
\mathcal{P}_{\mathcal{AB}}[z_{\mathcal{T}}] = h_{\mathcal{A}}(z_0)\mathcal{P}[z(\mathcal{T})]
h_{\mathcal{B}}(z_{\mathcal{T}})/Z_{\mathcal{AB}}(\mathcal{T})\label{eqn:tpsensem} \nonumber 
\end{equation}
where 
\[
    h_{\mathcal{A}/\mathcal{B}}(z)= 
\begin{cases}
    1, & \text{if } z\in \mathcal{A}/\mathcal{B}\\
    0,              & \text{otherwise}
\end{cases}
\]
and $Z_{\mathcal{AB}}(\mathcal{T})$ is the normalization factor for this 
probability distribution function.
  \end{block}

  \begin{block}{Committor analysis and transition state}
  \begin{figure}
        \subfloat[Committor]{\includegraphics[width=0.4\colwidth]{figures/separatrix-full.png}}
        \qquad
        \subfloat[Transition State]{\includegraphics[width=0.4\colwidth]{figures/dist120.pdf}}
        %\caption{The protein witht he substrates bound}
    \end{figure}

  \end{block}

  \begin{block}{Transition state structure of the MBHase enzyme}
  Calculations suggest that the Arg124 residue stabilizes the O$^{-}$ anion and is essential for catalysis 
  which is in agreement with experiments.  
\begin{figure}
\includegraphics[width=0.54\colwidth]{figures/trans-120.png}
\end{figure}
  \end{block}

\end{column}

\separatorcolumn

\begin{column}{\colwidth}

  

  \begin{block}{Free energies from TPS}
    A method to calculate the free energies within TPS was recently applied to enzyme catalyzed reactions. \cite{Balasubramani22JPhysChemB126p5413}
    \begin{figure}
        \centering
        \subfloat[Modified shooting algorithm]{\includegraphics[width=0.4\colwidth]{figures/tps-free.png}}
        \quad
        \subfloat[Calculated Free energy for reaction catalyzed by the human MAT2A enzyme]{\includegraphics[width=0.5\colwidth]{figures/mat2a-fenergy.pdf}}
        %\caption{Modified shooting algorithm to calculate free energies}
    \end{figure}
    \begin{equation}
A[\lambda_i] = -k_BTlog(P(\lambda_i)) +\;\text{constant} \nonumber 
\end{equation}
The probability of $P(\lambda_i)$ is defined by
\begin{equation}
P(\lambda_i) = \int \textbf{dV} \rho({\vec{q}})\delta(\lambda_i-\tilde{\lambda}(\vec{q})) \nonumber 
\end{equation}
In practice, the calculation of $P(\lambda_i)$ proceeds through histogram based techniques.
  \end{block}
\begin{block}{Conclusions and future directions}
\begin{itemize}
    \item The transition state structure for the reaction catalyzed by an MBHase enzyme was obtained
    \item The more catalytically efficient variants are currently under investigation and comparisons 
    would be drawn to the reaction mechanism using the TPS methods discussed here 
\end{itemize}
    
  \end{block}
  \begin{block}{References}

    \nocite{*}
    \footnotesize{\bibliographystyle{unsrtnat}\bibliography{poster}}
\begin{figure}
\centering
\includegraphics[scale=0.5]{figures/nih.png}
%\caption{}
\end{figure}
\emph{Research reported here was supported by National Institutes of Health under award number R35GM145213}
  \end{block}

%\begin{block}

%\end{block}

\end{column}

\separatorcolumn
\end{columns}
\end{frame}

\end{document}
